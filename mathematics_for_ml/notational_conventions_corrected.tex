\documentclass[11pt]{article}
\usepackage[T1]{fontenc}
\usepackage{lmodern}
\usepackage{microtype}
\usepackage{amsmath, amssymb, bm}
\usepackage[margin=1in]{geometry}
\usepackage{booktabs}

\title{Notational Conventions (Corrected)}
\author{}
\date{}

\begin{document}
\maketitle

\section*{Overview}
Unless noted otherwise, we use a superscript $i$ to denote the $i$th training example (row of the dataset) and a subscript $j$ to denote the $j$th feature (column of the dataset). Lowercase boldface letters denote vectors ($\bm{x} \in \mathbb{R}^{n \times 1}$). In other words, $\bm{x}$ is a column vector with $n$ components, each of which is a real number. Uppercase boldface letters denote matrices ($\bm{X} \in \mathbb{R}^{n \times m}$). Single elements are written in italics, e.g.\ $x^{(i)}$ or $x^{(i)}_j$.

\section*{Dataset, Rows (Samples), and Columns (Features)}
Let the dataset matrix be
\begin{equation*}
  \bm{X} \in \mathbb{R}^{150 \times 4},
\end{equation*}
where each of the $150$ rows is a flower instance and the $4$ columns are the features (e.g., sepal length, sepal width, petal length, petal width).

\paragraph{Row (sample) vector.}
The $i$th training example (the $i$th row of $\bm{X}$) is the row vector
\begin{equation*}
  \bm{x}^{(i)} \;=\; \begin{bmatrix} x^{(i)}_1 & x^{(i)}_2 & x^{(i)}_3 & x^{(i)}_4 \end{bmatrix}
  \;\in\; \mathbb{R}^{1 \times 4}.
\end{equation*}

\paragraph{Column (feature) vector.}
The $j$th feature across all examples (the $j$th column of $\bm{X}$) is the column vector
\begin{equation*}
  \bm{x}_{j} \;=\; \begin{bmatrix}
    x^{(1)}_{j} \\[2pt]
    x^{(2)}_{j} \\
    \vdots \\
    x^{(150)}_{j}
  \end{bmatrix}
  \;\in\; \mathbb{R}^{150 \times 1}.
\end{equation*}

\paragraph{Individual entry.}
The scalar $x^{(i)}_j$ denotes ``feature $j$ of training example $i$''. For instance, $x^{(150)}_1$ is the first feature (sepal length) of the $150$th example.

\section*{Targets / Labels}
The target variables (class labels) can be assembled into the column vector
\begin{equation*}
  \bm{y} \;=\; \begin{bmatrix}
    y^{(1)} \\
    y^{(2)} \\
    \vdots \\
    y^{(150)}
  \end{bmatrix},
  \qquad
  \text{where } y^{(i)} \in \{\text{Setosa}, \text{Versicolor}, \text{Virginica}\}.
\end{equation*}

\section*{Quick Reference}
\begin{center}
\begin{tabular}{@{}lll@{}}
\toprule
Object & Type / Shape & Meaning \\
\midrule
$\bm{X}$ & $150 \times 4$ matrix & Entire dataset \\
$\bm{x}^{(i)}$ & $1 \times 4$ row vector & Sample $i$ (features for one example) \\
$\bm{x}_j$ & $150 \times 1$ column vector & Feature $j$ across all examples \\
$x^{(i)}_j$ & scalar & Entry at row $i$, column $j$ \\
$\bm{y}$ & $150 \times 1$ column vector & Class labels \\
\bottomrule
\end{tabular}
\end{center}

\end{document}
